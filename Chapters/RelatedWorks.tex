\chapter{Related Work}
\label{chp:relatedwork}

In the article \cite{Tgwri1s2017} by Tgwrils (2017) we can further research into how the new rating system works and this gives us a better insight of how we can rate our metrics for the algorithm. The rating system is explained in the article as rating performance and not rating individual highlight moments. It focuses more on the overall performance of a player over a match. 
\\\\
In the paper \cite{Rubin2022} by Rubin, A. (2022, January 13). There are a lot of useful metrics and data we can use, such as what is the most significant factor for a team to win a round, Rubin presents data that show that the team's equipment value is one of the leading causes for a chance to win. This information we can take into account when creating our new algorithm. We can for example interpret this as if a player with poorer equipment value eliminates the enemy team, it is higher valued in the algorithm than if the player had more valuable equipment.\\\\ 
It is also stated how they analyze data from games and parse the demos they collected, which is of great use to us as we also need to collect similar data.\\\\
In the article "On broadcasted game video analysis: event detection, highlight detection, and highlight forecast" by Chu, W.-T. and Chou, Y.-C. (2016) they go over the topic of highlight detection which is appropriate for us. They go over metrics such as motion intensity, frame dynamics, number of gamers, event ratio, number of viewers chat and number of emotion symbols. All of these metrics might not be possible to collect as we are not analyzing a live event where there are any active viewers or interaction with any chat or crowd. As the article does however cover the topic of arousal of viewers, it is worth studying as the purpose of a highlight is to create excitement.\\\\
In the paper  "Automatic soccer video analysis and summarization," by m A. Ekin, A. M. Tekalp and R. Mehrotra (2003). We get to read on how a very different sport handles video analysis and summarization of games. The text is interested in the sense how other sports/activities can be analyzed and what to can be looked at, for example in the article there is mentioning of slow-motion segments which could also be used in highlights to show the event in more detail.
\\\\
In the master thesis "CS:GO Player Skill Prediction" \cite{BaranNama} we get to read about \todo{complete}
\\\\
The articles mentioned covers different aspects of analyzing games via metrics, this relates to our work as to find highlights we need to analyze the game similarly. 