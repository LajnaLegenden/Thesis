\chapter{Results and Analysis WHAT}
\label{chp:results}
\section{On the content}
\textbf{Our text:}\\
\section{Results from interviews}

The interviews were focusing on the metrics and what people liked about highlights. The interviews lay a baseline on what metrics should be included and metrics we already have included should be valued. The individuals we interviewed had different experience with highlights, Counter Strike 2 and other FPS games. Out of the 8 interviews we conducted, all had at least played one game of Counter Strike and knew of the game. Some with more experience than others. Many of the interviewees have a couple of thousand hours in the Counter Strike scene and are or have been invested in the game for years, while others have just a couple of games under their belt.\\\\
\subsection{First question}
The first question we asked the interviewees was, "What is your experience with Counter Strike or other FPS games?". From this question, we can have a broader aspect of what people with different experience think of such a topic as highlights in a game such as Counter Strike. The result we got from the question was that 6 out of 8 had a lot of experience in the game Counter Strike. While one interviewee had minimal experience with the game and another did not have much experience in Counter Strike but in other FPS games. One interviewee said, "I've been playing CS since I was a kid, more or less, 1.6. And then Source and all that. And then I started playing CS a lot in 2014. And have been playing CS since then every week. Since 2014 approximately. With some breaks here and there. But relatively regularly every year I played CS."\\\\
\subsection{Second question}
The second question we asked was "What is your experience with highlights in FPS games?". The result we got from this question was that most of the interviewees knew what they were, or with reference to other activities could understand the concept of a highlight. Many of the interviewees that had a lot of hours in Counter Strike had used programs such as Medal or similar to record the last 30 seconds or more to later edit or just save the clips. Others also answered with automatic clipping tools from sites such as Leetify and Esportal. Two out of the eight interviewees had no experience with highlights in FPS games.\\\\
\subsection{Third question}
The third question we asked, "What do you think is the most memorable highlight in Counter Strike history?". Here again the interviewees with a lot of hours in Counter Strike could reference highlights from the e-sport scene. Highlights that were mentioned was "happy 1Deag ace on Banana" \textbf{BENJAMIN REFRENCE} which happened during the game between the e-sport teams "EnVyUs vs TSM" where the player Happy eliminates the whole enemy team with the weapon Desert Eagle during DreamHack Open London 2015. Another interviewee mentions "After all, I have both famous ones from big tournaments or matches like Coldzera when he jumps and shoots on Mirage."\textbf{LIMPAN REFRENCE} which refences to a match between "Luminosity vs Liquid" during MLG Columbus 2016. The last refrence we get to e-sport matches is with an interviwee that says, "What I think is the best? Off the cuff, what pops into my head is Hiko. When he was clutching Dust 2 or something like that." Which references to the clip of the pro player Hiko getting a 180 degree flickshot on an enemy hiding behind a door on the map Dust2 during EMS One Katowice 2014 major.\\\\
The other interviewees answered with more general answers such as highlights that they had experience on their own, such as when they eliminated four players in a single round or when they eliminated the whole enemy team.\\\\
The follow-up question to the previous question is "What makes the highlight memorable?". Here we get interviewees referencing to casters of e-sports games "For me, that's what makes a good highlight. Precisely the part that lifts it is the skill demonstration. But also when the commentator is so damn hyped. That combination to me is absolutely fantastic. You get a lot of everything. If it's not from the clip, it's from the commentator. This for me means a lot. If it is no commentator. Which it is not when you play yourself. Then music can be the supporting part. That you have music that is timed with the shot or similar." \textbf{AIIMAR REFRENCE} another answered "The casters. Caster. Yeah. I suppose also the importance of the round". Others mentions the technical skills of the player, they won an impossible round or one answered "It must be something cool that happened. The more kills, the better. The faster the kills, the closer the kills are to each other, the better. The more shady the weapon, if you do something cool with it, the better." \textbf{WOTTOWMAN02 REFERENCE}\\\\
\subsection{Fourth question}
The fourth questions we asked was "What makes a good highlight?". From these questions we got a lot of different answers. One interviewee answered with "Quick succession. Quick responses. Like going from one to the other in rapid succession. And quantity of kills, usually." \textbf{BENJAMIN REFRENCE} while another answered with "I have a hard time pointing to a thing. But if I have to point to one thing, it's one taps. I think that is absolutely fantastic. I think that is always worth including in a highlight. Deagle OneTaps. It could be AKs. If you do it several times in the same round, or you have 5 kills via OneTaps." \textbf{AIIMAR REFERENCE}. These answers focus on the technical skill aspects of the highlights, while one interviewee said "A good highlight is made when there are a lot of people watching at the same time, when a good commentator at a professional match does a lot. And the circumstances, such as the thing that he is in Sweden and does such a thing against a non-Swedish team, means that everyone gets extra hyped."\textbf{zpam REFERENCE} about the pro-scene and on individual level said "Then it is probably important that you want to play with friends. So that they actually see you. And the best thing is if it's one against five or similar so that everyone can see you when you play." \textbf{zpam REFRENCE} focusing more on the social aspect of highlights and that it is something that gets the crowd or friends going. So from this we can gather that depending on the context of the game context and what scene players are playing in, there is a difference on what is considered highlight worthy.\\\\
\subsection{Fifth question}
Question five was about valuing metrics we provided to the interviewees from 1 to 10.
        \begin{itemize}
        \item Amount of kills: average = 8
        \item if the player died: average = 6,142857143
        \item Weapon used average: = 7,071428571
        \item Time between each kill: average = 7,142857143
        \item Bullets fired average: = 7,357142857
        \item Team side: average = 2,428571429
        \item Player position (Distance): average = 6,285714286
        \item Airborne average: = 8,285714286
        \item Headshot average: = 7,785714286
        \item Damage given average: = 5,928571429
        \item HP of the player average: = 6,714285714
        \item No scope with scoped weapons average: = 7,071428571
        \item Shooting through objects average: = 8,142857143
    \end{itemize}
Interesting takes from the average is that the team side has very low impact on a highlight.\\\\
\subsection{Sixth question}
Question six was, "is there any metric that we have forgotten?". On this question, one interviewee answered, "Something you didn't mention was this movement thing." which we did not include in the metrics but could be worth thinking about when developing the algorithm. One problem with accounting for movement is that it is hard to do, and what should be considered when looking at movement. Such, if a player moves fast between kills or if a player b-hops cross the map and eliminates an enemy. Another interviewee said, "The hit ratios are important to me. And maybe crosshair placement." Shots fired could be swapped out to hit ratio because as there could be situation where more bullets are necessary such as shooting through walls or where there is damage fall-off based on distance. Crosshair placement is also hard to weigh because of the implementation technicality, but it could be implemented. One interviewee said, "After all, it is to create a metric to start running highlights when enemies are close to each other. Type in smokes or distance wise. Because if you are very close to each other on a site or something, highlights are rarely created there. That they passed each other, were close to each other or that they did not see each other. Something along those lines.". To measure times when allies and enemies are in proximity to each other in a smoke or similar will also be hard to implement because of the technicality of the metric, but it could also be done. \\\\
\subsection{Seventh question}
Question seven "Do you use any tool to create highlights?" most interviewees answered that they had either used tools such as Medal, Faceit's highlight clipper, Esportal's highlight clipper. One interviewee answered, "I don't use anything. If I were to do it, I would use something that was simple and fast. Which fixed nice highlights. Which I could save and display. So, share those that might be able to be sent on discord directly. Or similar. like GIFS". So simplicity and portability would be important.\\\\
\subsection{Eight question}
The follow-up to the question above was "What do you do with the highlights?". Where some interviewees answered that they put them on YouTube in compilations with other clips, and some share them with friends. So highlights is mostly used to recap good moments and to be able to share them with friends and others.
\\\\
\subsection{Conclusion from interviews}
Based on the answers we got from the questions, we now have a baseline of what we should value the different metrics. We also got new thoughts on what people look for in highlights and got examples of highlights that the interviewees liked. During the interviews we also noticed a stark difference in the specificity of answers with the people who had more experience with Counter Strike than those without. 

