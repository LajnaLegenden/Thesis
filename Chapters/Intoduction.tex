\chapter{Introduction}
\label{chp:introduction}  % labels are used for cross references
\section{Background}
\todo{change abreviations or write like tournament change player nicknames}
In the world of Counter-Strike, there are a few moments that all players of the game are familiar with. These are often plays that work against all odds, with everything at stake. A few of these moments are when players such as Oleksandr Kostyljev also known as "S1mple", during the tournament ESL One New York 2016, throws his \gls{AWP}, the best weapon in the game to confuse his opponent on Dust 2, or the one where Russel Van Dulken also known as "Twistzz" pushes through the defense on the map Nuke during the tournament IEM Cologne 2022 or probably the most iconic of all, the player Marcelo David also known as "Coldzeras" B site hold on the map Mirage, at the MLG Columbus 2016 tournament. He single-handedly stopped Team Liquid from winning the match, with the score at 15-9 for Team Liquid. This, combined with the skill and luck involved, makes this probably the most recognized Counter-Strike highlight of all time.\\\\
Team Liquid is approaching bombsite B with all of its members through what is called apartments. Coldzera is holding this part of the map scoped in with his AWP. The first player leading the push gets spotted by Coldzera and gets eliminated. The four remaining members of Team Liquid starts pushing toward Coldzera as he repositions to a new spot. Coldzera then does the most unthinkable play in his position and jumps and fires without scoping his AWP, significantly decreasing the accuracy of the weapon. He eliminates two opponents with this risky play and with only two opponents left he finishes off the fourth opponent with a quick no-scope and his teammate TACO eliminates the last standing member of Team Liquid.\\\\
Thanks to this unbelievable play by Coldzera, he and his team Luminosity Gaming go on to win the tournament. Highlights like this are what most people watch competitive e-sports for, and when watching the highlight the crowd can be heard roaring through the arena and the casters praise\\\\
\todo{Never use "you" change all over}
\todo{change tense}
Most players playing Counter-Strike have these moments themselves, perhaps with less on the line, but the feeling of holding down a part of the map all alone is a great feeling. Some people compile all of their best moments into a highlight reel to upload for other people to see. Our question is, can we find these highlights by analyzing the match and finding better highlights for a given player? We are going to research if that is possible, how that could be done and if players opinions of what makes a good highlight differ depending on their knowledge of the game.\\\\
A highlight is when something extraordinary occurs during a time period or an event. In the case of Counter-Strike 2 a highlight could be a multitude of things, for example, a \gls{round} when a player kills multiple players in one round or a player kills more than one player with one bullet or a single \gls{utility}. \\\\
A lot of things occur during a Counter-Strike 2 match, meaning it can be hard to decide what highlight is better than another. In a single match, multiple highlight-worthy moments might happen, such as in one round a player might eliminate the whole enemy team, while in another round a player might kill three players with one grenade. This is where the algorithm comes in and values each significant play and decides which round would be better to make a highlight out of.


\section{Scope of the thesis}
This thesis studies the possibilities of automating the process of finding and selecting highlights in Counter-Strike 2.  Can we develop algorithms to analyze gameplay data, recognize impressive plays, and even rate them based on objective criteria?  Also, how do player opinions of what constitutes a "better" highlight differ based on their level of skill?  By researching these topics, we seek to discover new opportunities for improving the spectator experience, rewarding individual performance, and deepening our understanding of the components that contribute to a better highlight. 
\\\\
This thesis will only compare our algorithm to \gls{allstar}s algorithm, as that one is the only one that runs after a match is played. The other available (Faceit, GifYourGame) only use a threshold to save highlights and do not choose a play of the game.
\section{Outline}
The structure of the thesis:\\\\
\begin{itemize}
    \item Chapter 2 goes over related works and why they are relevant
    \item Chapter 3 goes over the method we used to develop our algorithm, interviews, and survey.
    \item Chapter 4 goes over the results from the interviews and survey as well as how one could create an algorithm.
    \item Chapter 5 presents the discussion of the results from the interviews, survey, and algorithm.
    \item Chapter 6 presents the conclusion and goes over any future work.
\end{itemize}
