
\chapter{Discussion}
\label{chp:discussion}
\section{Discussion interviews}
The interviews focused on the metrics and what people liked about highlights. It lays a baseline on what metrics should be included and what metrics we already have included should be valued. The individuals we interviewed had different experiences with highlights, Counter-Strike 2, and other FPS games. Out of the 8 interviews we conducted, all had at least played one game of Counter-Strike and knew of the game. Some with more experience than others. Many of the interviewees had a lot of experience with the Counter-Strike scene and are or have been invested in the game for years, while others have just a couple of games under their belt.\\\\
The first question we asked the interviewees was, "What is your experience with Counter-Strike or other FPS games?". From this question, we can have a broader aspect of what people with different experience think of such a topic as highlights in a game such as Counter-Strike. The result we got from the question was that 6 out of 8 had a lot of experience in the game Counter-Strike. While one interviewee had minimal experience with the game and another did not have much experience in Counter-Strike but in other FPS games. One interviewee said, "I've been playing CS since I was a kid, more or less, 1.6. And then Source and all that. And then I started playing CS a lot in 2014. And have been playing CS since then every week. Since 2014 approximately. With some breaks here and there. But relatively regularly every year I played CS."\\\\
The second question we asked was "What is your experience with highlights in FPS games?". The result we got from this question was that most of the interviewees knew what they were, or with reference to other activities could understand the concept of a highlight. Many of the interviewees who had a lot of hours in Counter-Strike had used programs such as Medal or similar to record the last 30 seconds or more to later edit or just save the clips. Others also answered with automatic clipping tools from sites such as Leetify and Esportal. Two out of the eight interviewees had no experience with highlights in FPS games.\\\\
The third question we asked was, "What do you think is the most memorable highlight in Counter-Strike history?". Here again, the interviewees with a lot of hours in Counter-Strike could reference highlights from the e-sport scene. Highlights that were mentioned was "happy 1Deag ace on Banana" (Participant B1, personal communication, April 21, 2024) which happened during the game between the e-sport teams "EnVyUs vs TSM" where the player Happy eliminates the whole enemy team with the weapon Desert Eagle during DreamHack Open London 2015. Another interviewee mentions, "After all, I have both famous ones from big tournaments or matches like Coldzera when he jumps and shoots on Mirage." 
(Participant L1, personal communication, April 10, 2024) which references a match between "Luminosity vs Liquid" during MLG Columbus 2016. The last reference we get to e-sport matches is with an interviewee who says, "What I think is the best? Off the cuff, what pops into my head is Hiko. When he was clutching Dust 2 or something like that." This references to the clip of the pro player Hiko getting a 180-degree flickshot on an enemy hiding behind a door on the map Dust2 during EMS One Katowice 2014 major.\\\\ 
What we can take away from this is that experienced players could give more concrete examples of highlights, but those with less knowledge could still give a vague answer of scenarios where a highlight would be applicable. 
The other interviewees answered with more general answers such as highlights that they had experienced on their own, such as when they eliminated four players in a single round or when they eliminated the whole enemy team.\\\\
The follow-up question to the previous question is "What makes the highlight memorable?". Here we get interviewees referencing casters of e-sports games "For me, that's what makes a good highlight. Precisely the part that lifts it is the skill demonstration. But also when the commentator is so damn hyped. That combination to me is absolutely fantastic. You get a lot of everything. If it's not from the clip, it's from the commentator. This for me means a lot. If it is no commentator. Which it is not when you play yourself. Then music can be the supporting part. That you have music that is timed with the shot or similar." (Participant A1, personal communication, April 09, 2024) another answered "The casters. Caster. Yeah. I suppose also the importance of the round". Others mentions the technical skills of the player, they won an impossible round or one answered "It must be something cool that happened. The more kills, the better. The faster the kills, the closer the kills are to each other, the better. The more shady the weapon, if you do something cool with it, the better." (Participant O1, personal communication, April 30, 2024). An important thing to notice from these answers is that many mentioned casters as an important part of highlights, especially in the e-sport scene. This is a valid input, but not in the scope of our proof of concept.\\\\
The fourth question we asked was "What makes a good highlight?". From these questions, we got a lot of different answers. One interviewee answered with "Quick succession. Quick responses. Like going from one to the other in rapid succession. And quantity of kills, usually." (Participant B1, personal communication, April 21, 2024) while another answered with "I have a hard time pointing to a thing. But if I have to point to one thing, it's one taps. I think that is absolutely fantastic. I think that is always worth including in a highlight. Deagle OneTaps. It could be AKs. If you do it several times in the same round, or you have 5 kills via OneTaps." (Participant A1, personal communication, April 09, 2024). These answers focus on the technical skill aspects of the highlights, while one interviewee said "A good highlight is made when there are a lot of people watching at the same time, when a good commentator at a professional match does a lot. And the circumstances, such as the thing that he is in Sweden and does such a thing against a non-Swedish team, means that everyone gets extra hyped."(Participant Z1, personal communication, April 26, 2024) about the pro-scene and on individual level said "Then it is probably important that you want to play with friends. So that they actually see you. And the best thing is if it's one against five or similar so that everyone can see you when you play." (Participant Z1, personal communication, April 26, 2024) focusing more on the social aspect of highlights and that it is something that gets the crowd or friends going. So from this, we can gather that depending on the context of the game context and what scene players are playing in, there is a difference in what is considered highlight-worthy.\\\\
Question five was about valuing metrics we provided to the interviewees from 1 to 10. An interesting take from the average is that the team side has a very low impact on a highlight.\\\\
Question six was, "Is there any metric that we have forgotten?". On this question, one interviewee answered, "Something you didn't mention was this movement thing." which we did not include in the metrics but could be worth thinking about when developing the algorithm. One problem with accounting for movement is that it is hard to do, and what should be considered when looking at movement. Such, as if a player moves fast between kills or if a player b-hops across the map and eliminates an enemy. Another interviewee said, "The hit ratios are important to me. And maybe crosshair placement." Shots fired could be swapped out to hit ratio because as there could be situations where more bullets are necessary such as shooting through walls or where there is damage fall-off based on distance. Crosshair placement is also hard to weigh because of the implementation technicality, but it could be implemented. One interviewee said, "After all, it is to create a metric to start running highlights when enemies are close to each other. Like in smokes or distance-wise. Because if you are very close to each other on a site or something, highlights are rarely created there. That they passed each other, were close to each other or that they did not see each other. Something along those lines.". Measuring times when allies and enemies are in proximity to each other in a smoke or similar will also be hard to implement because of the technicality of the metric, but it could also be done. \\\\
Question seven "Do you use any tool to create highlights?" most interviewees answered that they had either used tools such as Medal, Faceit's highlight clipper, or Esportal's highlight clipper. One interviewee answered, "I don't use anything. If I were to do it, I would use something that was simple and fast. Which fixed nice highlights. Which I could save and display. So, share those that might be able to be sent on Discord directly. Or similar. like GIFs". So simplicity and portability would be important.\\\\
The follow-up to the question above was "What do you do with the highlights?". Where some interviewees answered that they put them on YouTube in compilations with other clips, and some shared them with friends. So highlights are mostly used to recap good moments and to be able to share them with friends and others.\\\\
Based on the answers we got from the questions, we now have a baseline of what we should value the different metrics. We also got new thoughts on what people look for in highlights and got examples of highlights that the interviewees liked. During the interviews, we also noticed a stark difference in the specificity of answers with the people who had more experience with Counter-Strike than those without. 
\section{Discussion on the algorithm and survey}
As we tried to make an algorithm that outputted better highlights, based on the results we got from the survey, our algorithm managed to get the slight edge as it was preferred 6 out of 9 times. As Allstars algorithm is not public, it is not possible to explain what metrics are focused on, and hard to do a full-on comparison against our algorithm. The information we get is the outputted round by Allstars. To understand why our video clips were preferred, we need to look at the difference between the videos and the answers the respondents gave. The matches where there was a major preference to one of the videos were matches 2, 3, 5, 7, 8, and 9. Worth mentioning is that out of 19 matches, 9 times Allstar and our algorithm outputted the same round. This seems to happen when a player only gets one round with 4+ kills and the rest are 1 and 2 kills. This could simply be that the player had one very good round and the remaining rounds were poorly played. This could mean the more similar rounds, the more difference in what our algorithm will pick and what Allstar will pick. 
\subsection*{Match 2}
In match 2 both clip A and B included 2 kills, but the winning clip B included a kill with the knife which was liked by the respondents. There were also comments that clip A was messy. It is however important to remember that our algorithm does not account for things such as movement or crosshair placement, which could be considered messy. We do not know if this is something that Allstar takes into account.
\subsection*{Match 3}
In match 3 we have the clips, A with 4 kills and B with 3 kills. A key difference in clip B the player did not die at the end of the clip which our algorithm takes into account. Even though in clip A which is from Allstar had 4 kills, the majority picked clip B with only 3 kills. One response that preferred our video B was "In video A the player dies, I do not think that is a good end of a highlight" So here our check for whether the player died mattered. Another respondent answered with "Video A was a nice 4K with a 2 kill spray down. Video B was more average" and seemed to value the 4th kill and a multi-kill highly, which gets lowered in our algorithm because of the death at the end of the clip. A discussion could also be held again for movement as one player wrote "Better checks on corners and utilized utilities and safe playmaking" we did not check for that in our algorithm, but our clip was still chosen.
\subsection*{Match 5}
In match 5, clip A contains 2 kills, and clip B contains 3 kills. In this match the player dies in clip A and not in clip B which two respondents answered with "In video B he does not die, and gets most of the kills" and "He didn't die, and won round". Even though there might need to be more data collected, the effect of dying seems to be important to players and our algorithm takes this into account. 
\subsection*{Match 7}
In match 7, clip A contains 2 kills and 1 team kill, and clip B contains 2 kills. The player does not die in any of the clips. The majority prefers clip B which was from our algorithm. The respondent's reasoning for this was a bit of spread, as one said "More kills, as well as a nice 2K spraydown. First shot was especially clean" and "I bet your algorithm is counting team kills as normal kills" even though the last kill was team kill they still picked clip A. Our algorithm does not count team kills as kills. The only respondent that picked clip B and had a reasoning said "Just a better round, he knows what he is doing and plays around the enemies making mistakes". There could be a case for the succession of kills in clip B was faster than in clip A but no comments about this were made.
\subsection*{Match 8}
Match 8 had two highlights. Highlight A contained a nice clutch with 3 kills, and highlight B had a good Glock round with two kills. Here highlight A won because of the clutch, and it had more and cleaner kills. The amount of kills is something that Allstar at least counts, and we can assume that is taken into account, since the title of their clips have the amount of kills in them, as well as the weapon that was used. Since clean came up a few times, this might relate to the accuracy, and that does our algorithm consider. In highlight B the opponent missed a few of his shots, only showcasing that the player should have died, but only survived to clutch the round because of the bad play by his opponent.
\subsection*{Match 9}
In match 9, highlight B includes three quick kills in the beginning, with a nice clutch at the end, showing off the game sense of the player, tricking the player into thinking that he had gone back to a different site, while in highlight A, there are three nice kills, two that are through a smoke. Here the respondents said that they liked highlight B better, since it was more tactical, and the Glock kills were nice. The two fast Glock kills is something that our algorithm takes into account, but in this case, highlight B was not chosen by our algorithm, and instead chosen by Allstar. We can not know if this is something that they are taking into account, but our algorithm takes both weapons and time between kills into account. Here the AK47 kills got chosen since they are a more difficult weapon. Both highlights are fast-paced, and as we can see, the weapons weights might need to be adjusted to make a highlight like this be chosen by our algorithm. Currently, the weapons might be slightly too much of a factor. \\\\

When analyzing the difference between what a novice player and an experienced player find enjoying in a highlight, we got little data to work with and could not find any patterns due to the lack of novice players in the survey pool. However, here are some finds from the novice players in the survey:

The first person mentioned in Match 1 that they liked that video b had more action in it, than video A. Video A consists largely of the player standing around and playing his spot until he decides to get and help out his team. This shows that some less familiar players like the action that you can see, and might not understand the deeper tactics of the game. None of the players left a comment on Match 8 but preferred Video A, where the players go onto the bombsite. However, in video B, the player runs back and forth, tricking the enemy that he has gone back to another bombsite. This video won, but the less familiar players choose Video A, since they might not see the tactics at play, and that he faked out his enemy.

The other respondent that said they were not familiar with the game said that they liked video A for its clutch, however, since they knew the terminology, they might be still somewhat familiar with the game, showing that people that have played the game appreciate the clutch and smart plays. This respondent did prefer Video A in match 8, so might just prefer the higher speed of the game in Video A.






